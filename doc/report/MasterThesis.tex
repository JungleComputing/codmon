\documentclass[a4paper,10pt]{scrartcl}
\usepackage[utf8]{inputenc}

% Title Page
\title{Codmon: A multi-platform modular test environment.}
\author{Berend van Veenendaal}
\bibliographystyle{plain}
%hieronder wordt de project naam als variabele gedeclareerd
\newcommand{\project}{Codmon 2.0}
\newcommand{\CS}{C\nolinebreak\hspace{-.05em}\raisebox{.6ex}{\bf \#}}

\begin{document}
\maketitle

\begin{abstract}
TODO:Abstract
\end{abstract}
\newpage
\section*{Preface}
\label{sec:Preface}
TODO: Preface,acknowledgements
\newpage
\tableofcontents
\newpage

\section{Introduction}
\label{sec:Introduction}
This chapter will give an introduction about the \project{} project by giving a brief decription of background of my research and of the
previous version of the Codmon project. It also describes the structure of the reminder of this thesis.

\subsection{Background}
\label{sec:Background}
In times when software projects become more and more complex, testing of this software becomes more and more important. Many software
related problems are caused by lack of testing of the software~\cite{TTCST}. One of the challenges of software engineering is to make
sure that the software behaves in the same way on different platforms. Even when software is written in such a way that it can run on multiple platforms, there 
are still issues that must be dealt with, before one is able to run and test the software. Think about the configuration of the test environment 
or finding and installing all the prerequisite libraries etc etc. 

\subsection{Problem indication}
\label{subsec:Problemindication}
TODO paper references for Hudson, Jenkins, Junit and CsUnit 
TODO Expand problem indication

Now days there are numerous test frameworks and test environments available. For example there is \emph{Junit} for Java-unit testing and \emph{csUnit} for \CS{}-unit testing.
There are also different environments like Hudson, Jenkins which can build a project and run a series of (unit) tests against this project. All of the frameworks
and environments have bother there advantages and disadvantages. One of the advantages of unit testing is that a software developer easily can add new \emph{functinal} unit tests.
One of the disadvantages is that standard unit testing ignores non-funtional tests like performance testing. Jenkins and Hudson,like Unit tests, also have their
disadvantages. One of them is, althoug they both run on different platforms, in their usage they are not realy platform independent. For example, If you want
to make a connection form Hudson or Jenkins to a remote machine you need to know in advance on which platform this remote machine runs.


\subsection{Problem statement}
\label{subsec:Problemstatement}
The test frameworks and test environments described in section \ref{subsec:Problemindication} can be criticized on one or more aspects. What we are looking for is, infact, a combination
of the positive aspects of the described frameworks and environments, without getting the undesirable aspects for free. So the big question is, is it possible to design a
multi-platform, modular test environment? If we manage to design such a test environment, is it also possible to design is in such a way that is user friendly and maintainable?
Whith user friendly we mean that is must be possible to add both new test cases and software under test to the test environment.


 This thesis describes a multi-platform modular test environment called \project{}. The \project{} project provides a set of virtual machines,
in which \project{} is aleady installed and preconfigured.

First, to run on multiple platforms the software must be written in a
platform independent language, for example Java~\cite{Java}.


TODO: Describe the problem (refer to other environments/platforms)
TODO: describe research questions.

\subsection{Thesis outline}
\label{subsec:Thesisoutline}
TODO: Describe whats where in this thesis

\newpage

\section{Codmon}
\label{sec:codmon}

\subsection{the road to codmon}
//TODO: Think of better subsection title!!
//TODO: Explain ideas and road to solution

\subsection{Codmon 2.0}
//TODO: adapt Codmon 2.0 to new project name.
//TODO: 2) Explain Solution, including why it's different then existing solutions (compare with solutions described in the problem indication)

\newpage
\section{Conclusion}
\subsection{conclsuion}
TODO: give answers to the questions from section Problemstatement
TODO: Discus related future work
\newpage

\bibliography{Master}
\end{document}    